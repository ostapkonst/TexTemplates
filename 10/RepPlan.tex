\documentclass[a4paper,12pt]{article} 

\usepackage[T2A]{fontenc}
\usepackage[utf8x]{inputenc} % Включаем поддержку UTF8
\usepackage[english,russian]{babel}
\usepackage{indentfirst} % включить отступ у первого абзаца

\usepackage{graphicx}

\usepackage{listings}

\usepackage[strict]{changepage}

\evensidemargin=4.6mm %при двустороннем наборе задает размеры левого поля для четных страниц
\oddsidemargin=4.6mm %величина левого поля при одностороннем наборе
\topmargin=-1in %высота верхнего поля
\textwidth=16cm %ширина текста
\textheight=277mm %высота текста
\begin{document} 

\begin{titlepage}
\sffamily \center \Large
	Федеральное агентство по образованию РФ\\
	Дальневосточный государственный университет\\
	Институт математики и компьютерных наук\\
	Кафедра информатики\\[7cm]
\Large
	Векторный графический редактор\\[7cm]
\Large
\begin{flushright}
	Практическая работа\\
	студента Б8103-A группы\\
	Константинова Остапа Владимировича\\
	Научный руководитель:\\
	Кленин Александр Сергеевич\\
\end{flushright}
\vfill
Владивосток, 2013
\end{titlepage}

\tableofcontents
\newpage

\section{Аннотация}

В настоящем документе описаны основные этапы создания векторного графического редактора от постановки задачи до непосредственной реализации оной в соответствии с документацией [1],[2]. 


\section{Введение}

Для создания и редактирования изображений в настоящее время в основном используется графические редакторы.

Целью практической работы является разработка программы «Графический редактор», которая позволяет пользователю просматривать и редактировать графические файлы.

Из поставленной цели вытекают следующие задачи, которые необходимо решить для разработки данного приложения:

\begin{itemize}
	\item  изучить методическую литературу по технологии разработки программных продуктов;
	\item  изучить теоретические аспекты среды программирования Lazarus;
	\item  подготовить дизайн-проект приложения;
	\item  организовать удобную навигацию в программе.
\end{itemize}


\subsection{Описание предметной области}

Программный продукт может быть использован для создания различных графических изображений, при помощи заданного набора примитивов, и позволяет преобразовывать полученное изображение. Программа разработана в учебных целях и не претендует на использование в качестве полноценного редактора.


\subsection{Неформальная постановка задачи}

Программный продукт предназначен для работы с изображением и поэтому должен позволять выполнять следующие действия:

\begin{itemize}
	\item  Работать с графическими файлами;
	\item  Создавать новые графические файлы;
	\item  Рисовать основные геометрические примитивы;
	\item  Иметь дополнительные возможности редактирования;
	\item  Сохранять графические файлы. 
\end{itemize}


\subsection{План работ}
17.11.2012 -- Добавить простые инструменты;

24.11.2012 -- Добавить возможность регистрации инструментов;

01.12.2012 -- Добавить возможность редактировать параметры фигур;

08.12.2012 -- Добавить палитру;

15.12.2012 -- Добавить масштабирование и прокрутку;

22.12.2012 -- Добавить возможность редактирование формы фигур;

29.12.2012 -- Добавить возможность сохранения и загрузки.


\section{Требования к окружению}


\subsection{Требования к аппаратному обеспечению}

\begin{tabular}{|p{1.3in}|p{1.6in}|p{1.5in}|} \hline 
\textbf{} & \textbf{Минимальные} & \textbf{Рекомендуемые} \\ \hline 
\textbf{Процессор} & 233 МГц & 300 МГц или выше \\ \hline 
\textbf{Оперативная память} & 64 МБ  & 128 МБ или выше \\ \hline 
\textbf{Свободное место на жёстком диске} & \multicolumn{2}{|p{3.1in}|}{30MB или больше} \\ \hline 
\end{tabular}


\subsection{Требования к программному обеспечению}

На компьютере пользователя должна быть установлена любая операционная система линейки Windows NT.


\subsection{Требования к пользователям}

Пользователь должен иметь опыт использования прикладными программами на компьютере под управлением, требуемой для работы приложения, операционной системы.


\section{Архитектура системы (Общие требования)}

Программа распространяется в качестве exe файла, и не требует каких - либо сторонних библиотек или же программ.


\section{Спецификация данных}

Для сохранения изображения и его последующей загрузке используется формат файлов с расширением .lpf.


\subsection{Описание формата или структуры данных}

Формат lpf:

\includegraphics*[scale = 0.60]{image1.jpg}

Рис. 1 Графическое изображение, описанное в примере.

Созданные файлы хранятся в формате lpf. В примере ниже описана структура сохранения Рис. 1 в формате векторного редактора.

\begin{lstlisting}
Vector Editor by LifePack
  Scale = 1
  OffsetX = 0
  OffsetY = 0
  FilledAreaLeft = 0
  FilledAreaTop = 0
  FilledAreaRight = 751
  FilledAreaBottom = 688
    figure = TEllipse
      PointsArr = |368/254| ,|518/405| 
      PC_PenCol = 255
      PW_PenWid = 0
      PS_PenSty = 0
      SelectedFigure = DrawFigure
      BC_BrushCol = 65535
      BS_BrushSty = 0
    end
    figure = TRoundRect
      PointsArr = |82/261|, |335/399|
      PC_PenCol = 65280
      PW_PenWid = 0
      PS_PenSty = 0
      SelectedFigure = DrawFigure
      BC_BrushCol = 33023
      BS_BrushSty = 0
      PW_WidthEll = 10
      PW_HeightEll = 10
    end
    figure = TLine
      PointsArr = |82/81| , |470/241| 
      PC_PenCol = 0
      PW_PenWid = 0
      PS_PenSty = 0
      SelectedFigure = DrawFigure
    end
\end{lstlisting} 


\section{Функциональные требования}

Система должна позволять пользователю:

\begin{itemize}
	\item  Работать с графическими файлами;
	\item  Создавать новые графические файлы;
	\item  Рисовать основные геометрические примитивы;
	\item  Сохранять графические файлы;
	\item  Обрабатывать ошибки файлов;
	\item  Редактировать свойства фигуры;
	\item  Редактировать свойства нескольких фигур одновременно;
	\item  Копировать, удалять, вставлять фигуры;
	\item  Перемещать фигуры по холсту;
	\item  Редактировать вершины фигур.
\end{itemize}




\section{Требования к интерфейсу}

Система должна:

\begin{itemize}
	\item  обладать дружественным интерфейсом;
	\item  иметь прозрачную архитектуру для пользователя, пользователю должна быть предоставлена возможность менять вид программы по своему усмотрению;
	\item  Функциональные элементы меню должны находится в левом и верхнем углах экрана, т.к. там они легче всего воспринимаются;
	\item  Пункты меню должны отражать основные моменты пользования программой.
\end{itemize}


\section{Проект}

Продукт состоит из 13 модулей;

2-х форм;

1 одного проекта;

папки Images.


\subsection{Средства реализации}

По требованию заказчика выбран язык Delphi, а работа проводилась в среде разработки Lazarus.


\subsection{Структуры данных}

Основная диаграмма классов - модуль Ufigures.

\includegraphics*[scale = 0.60]{image2.jpg}


\subsection{Модули и алгоритмы}

Uabout -- Форма ``О программе'';

Uedialoglist -- Форма ``Окно об ошибках'';

Uedits -- Модуль работы с редакторами свойств;

Uditsstatic-- Модуль работы с редакторами свойств для статических объектов;

Uexceptions -- Описывает типы ошибок;

Ufigures-- Модуль описывает и реализовывает фигуры;

Ugraph-- Модуль реализовывает работу с некоторыми переменными;

Umain-- Главная форма;

Unotvisualtools-- Не визуальные инструменты;

Urtti-- Работа с RTTI;

Usaveloadtolpf-- Загрузка и сохранение информации;

Utransformation-- Трансформирование;

Uvisualtools-- Визуальные инструменты.


\subsection{Стандарт кодирования}

Использован стандарт кодирования, принятый сообществом Lazarus. 


\subsection{Проект интерфейса}

Основные элементы интерфейса:

\begin{flushleft}
\includegraphics*[scale = 0.60]{image3.jpg}

	Рис. 2 Основное окно программы

\includegraphics*[scale = 0.60]{image4.jpg}

	Рис. 3 Набор инструментов

\includegraphics*[scale = 0.60]{image5.jpg}

	Рис. 4 Редактор свойств

\includegraphics*[scale = 0.60]{image6.jpg}

	Рис. 5 Редактор цветов
\end{flushleft}


\section{Заключение}

Таким образом, в процессе работы мною был сделан графический векторный редактор, за который я получил зачет, а также повысил свои навыки программирования.


\section{Список литературы}

\begin{thebibliography}{10}

\bibitem{Klenin1}{Кленин А. С.} \emph{Методические указания по подготовке и защите отчётов}, ДВГУ\\ \url{http://imcs.dvgu.ru/ru/courses/repplan}

\bibitem{Klenin2}{Кленин А. С.} \emph{Технология про граммирования: программа курса}, ДВГУ\\
\url{http://imcs.dvgu.ru/ru/courses/progtech}
\end{thebibliography}

\end{document}

